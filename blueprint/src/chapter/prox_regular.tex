\chapter{Prox-Regular Case}

This chapter establishes convergence of the catching-up algorithm for uniformly prox-regular moving sets.

\section{Uniformly Prox-Regular Sets}

\begin{definition}[Uniformly prox-regular set]
\label{def:prox_regular}
\uses{def:proximal_normal_cone}
\lean{MoreauSweeping.UniformlyProxRegular}
\leanok
Let $S \subset H$ be a closed set and $\rho \in (0, +\infty]$. The set $S$ is called $\rho$-uniformly prox-regular if for all $x \in S$ and $\zeta \in N^P(S; x)$ one has
$$\langle \zeta, x' - x \rangle \le \frac{\|\zeta\|}{2\rho} \|x' - x\|^2$$
for all $x' \in S$.
\end{definition}

\begin{proposition}[Characterization of prox-regularity]
\label{prop:prox_regular_characterization}
\uses{def:distance,def:enlargements,def:prox_regular,def:proximal_normal_cone}
\lean{MoreauSweeping.prox_regular_characterization}
\leanok
Let $S \subset H$ be a closed set and $\rho \in ]0, +\infty]$. The following assertions are equivalent:
\begin{enumerate}
\item $S$ is $\rho$-uniformly prox-regular
\item For any $\gamma \in (0,1)$, the projection map $\text{proj}_S$ is well-defined on $U_\rho^\gamma(S)$ and Lipschitz continuous with constant $(1-\gamma)^{-1}$
\item For any $x_i \in S$, $v_i \in N^P(S; x_i)$ with $i = 1, 2$:
$$\langle v_1 - v_2, x_1 - x_2 \rangle \ge -\frac{1}{2\rho} (\|v_1\| + \|v_2\|) \|x_1 - x_2\|^2$$
\item For all $\gamma \in ]0,1[$, for all $x, x' \in U_\rho^\gamma(S)$, and for all $\xi \in \partial_P d_S(x)$,
$$\langle \xi, x' - x \rangle \le \frac{1}{2\rho(1-\gamma)^2}\|x' - x\|^2 + d_S(x') - d_S(x).$$
\end{enumerate}
\begin{proof}
This is the constant-radius specialization of the equivalences proved by Colombo--Thibault for variable radius $\rho(\cdot)$ (their Theorem~3).

\smallskip
\noindent\emph{(1) $\Rightarrow$ (2).}
If $S$ is $\rho$-uniformly prox-regular, then Theorem~3(f) with $\rho(\cdot)\equiv\rho$ yields: for each $\gamma\in(0,1)$, the projection is single-valued on $U_\rho^\gamma(S)$ and
$$
\|\text{proj}_S(u_1)-\text{proj}_S(u_2)\|\le (1-\gamma)^{-1}\|u_1-u_2\|.
$$
Hence (2) holds.

\smallskip
\noindent\emph{(2) $\Rightarrow$ (1).}
Take $x\in S$, $v\in N^P(S;x)$ with $\|v\|\le 1$, and $t\in(0,\rho)$. To prove $\rho$-prox-regularity it is enough to show
$$
x\in \text{Proj}_S(x+tv).
$$
Choose $t<t'<\rho$ and set $\gamma=t'/\rho\in(0,1)$. Then
$$
d_S(x+tv)\le \|x+tv-x\|=t<\gamma\rho,
$$
so $x+tv\in U_\rho^\gamma(S)$ and $\text{proj}_S(x+tv)$ is well-defined by (2). Using the projection/resolvent identification for the proximal normal cone from Theorem~3 (constant-radius case), one gets $\text{proj}_S(x+tv)=x$. Therefore $x\in\text{Proj}_S(x+tv)$, i.e. $S$ is $\rho$-uniformly prox-regular.

\smallskip
\noindent\emph{(1) $\Leftrightarrow$ (3).}
In Theorem~3, prox-regularity is equivalent to the hypomonotonicity inequality for proximal normals. For constant radius, that inequality becomes exactly
$$
\langle v_1-v_2,x_1-x_2\rangle
\ge -\frac{1}{2\rho}(\|v_1\|+\|v_2\|)\|x_1-x_2\|^2,
$$
which is (3).

\smallskip
\noindent\emph{(2) $\Rightarrow$ (4).}
Fix $\gamma\in(0,1)$ and $x\in U_\rho^\gamma(S)$, and write $p=\text{proj}_S(x)$. By (2), $\text{proj}_S$ is Lipschitz on the tube, which yields $C^{1,1}$ regularity of $x\mapsto \frac12 d_S(x)^2$ there, with modulus controlled by $(1-\gamma)^{-1}$. Translating this estimate to $d_S$ via proximal subgradients gives
$$
d_S(x')\ge d_S(x)+\langle\xi,x'-x\rangle - \frac{1}{2\rho(1-\gamma)^2}\|x'-x\|^2,
$$
equivalent to (4).

\smallskip
\noindent\emph{(4) $\Rightarrow$ (3).}
The semiconvexity-type estimate in (4) implies a near-monotonicity inequality for $\partial_P d_S$. Passing to the limit along proximal normal rays approaching $S$ recovers the hypomonotonicity inequality for $N^P(S;\cdot)$ in (3).

Combining the implications,
$$
(4)\Rightarrow(3)\Leftrightarrow(1)\Leftrightarrow(2),
$$
so all four statements are equivalent.
\end{proof}
\end{proposition}

\begin{remark}
Convex sets are $\rho$-uniformly prox-regular for any $\rho > 0$. This class also includes nonconvex bodies with $C^2$ boundary.
\end{remark}

\section{Properties of Approximate Projections for Prox-Regular Sets}

\begin{proposition}[Convergence of approximate projections]
\label{prop:approx_proj_convergence}
\uses{def:approximate_projection,lem:approximate_projection_formula,prop:prox_regular_characterization}
\lean{MoreauSweeping.approx_proj_convergence_prox_regular}
\uses{def:approximate_projection, lem:approximate_projection_formula}
Let $S \subset H$ be $\rho$-uniformly prox-regular. If $(x_n) \to x \in U_\rho(S)$ and $z_n \in \text{proj}_S^{\varepsilon_n}(x_n)$ with $\varepsilon_n \to 0$, then $z_n \to \text{proj}_S(x)$.
\end{proposition}

\begin{proposition}[Quasi-Lipschitz property]
\label{prop:quasi_lipschitz}
\uses{def:approximate_projection,lem:approximate_projection_formula,prop:prox_regular_characterization}
\lean{MoreauSweeping.quasi_lipschitz_approx_proj}
\uses{def:approximate_projection, prop:approx_proj_convergence}
For suitable $\gamma$ and $\varepsilon$, approximate projections on prox-regular sets satisfy a quasi-Lipschitz property relating the distance between projections to the distance between original points.
\end{proposition}

\section{Convergence Result}

\begin{theorem}[Convergence for prox-regular sets]
\label{thm:convergence_prox_regular}
\uses{thm:algorithm_properties,eq:sweeping_process,def:hausdorff,prop:prox_regular_characterization,prop:approx_proj_convergence,prop:quasi_lipschitz}
\lean{MoreauSweeping.convergence_prox_regular}
\leanok
\uses{thm:algorithm_properties, prop:approx_proj_convergence, prop:quasi_lipschitz}
Let $(C(t))_{t\in[0,T]}$ be a family of uniformly prox-regular sets. Under appropriate assumptions on the discretization and error sequence, the sequence $(x^n)$ generated by the catching-up algorithm converges to the unique solution of the sweeping process.
\end{theorem}
