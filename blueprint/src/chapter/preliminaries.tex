\chapter{Preliminaries}

This chapter introduces the mathematical preliminaries needed for the development of the catching-up algorithm with approximate projections.

\section{Basic Definitions}

Throughout this work, $H$ denotes a real Hilbert space with norm $\|\cdot\|$ induced by an inner product $\langle \cdot, \cdot \rangle$.

\begin{definition}[Distance function]
\label{def:distance}
For a set $S \subset H$, the distance function is defined as
$$d_S(x) := \inf_{z \in S} \|x - z\|$$
for all $x \in H$.
\end{definition}

\section{Support and Enlargements}

\begin{definition}[Support function]
\label{def:support}
For a set $S \subset H$, the support function at $x \in H$ is defined as
$$\sigma(x, S) := \sup_{z \in S} \langle x, z \rangle$$
\end{definition}

\begin{definition}[Enlargements]
\label{def:enlargements}
Given $\rho \in ]0, +\infty]$ and $\gamma < 1$ positive, the $\rho$-enlargement and $\gamma\rho$-enlargement of $S$ are defined as
$$U_{\rho}(S) = \{x \in H : d_S(x) < \rho\}$$
$$U_{\rho}^{\gamma}(S) := \{x \in H : d_S(x) < \gamma \rho\}$$
\end{definition}

\begin{definition}[Hausdorff distance]
\label{def:hausdorff}
The excess of $A$ over $B$ is $e(A, B) := \sup_{x \in A} d_B(x)$. The Hausdorff distance is
$$d_H(A, B) := \max\{e(A, B), e(B, A)\}$$
\end{definition}

\section{Normal Cones}

\begin{definition}[Clarke tangent cone]
\label{def:clarke_tangent_cone}
A vector $h \in H$ belongs to the Clarke tangent cone $T(S; x)$ when for every sequence $(x_n)$ in $S$ converging to $x$ and every sequence of positive numbers $(t_n)$ converging to $0$, there exists a sequence $(h_n)$ converging to $h$ such that $x_n + t_n h_n \in S$ for all $n \in \mathbb{N}$.
\end{definition}

\begin{definition}[Clarke normal cone]
\label{def:clarke_normal_cone}
The Clarke normal cone to $S$ at $x \in S$ is defined as
$$N(S; x) := \{ v \in H : \langle v, h \rangle \le 0 \text{ for all } h \in T(S; x) \}$$
where $T(S; x)$ is the Clarke tangent cone.
\end{definition}

\begin{definition}[Proximal subdifferential]
\label{def:proximal_subdifferential}
Let $f : H \to \mathbb{R} \cup \{+\infty\}$ be lower semicontinuous and $x \in \text{dom } f$. An element $\zeta$ belongs to the proximal subdifferential of $f$ at $x$, denoted $\partial_P f(x)$, if there exist $\sigma, \eta \geq 0$ such that
$$f(y) \ge f(x) + \langle \zeta, y - x \rangle - \sigma \|y - x\|^2$$
for all $y \in \mathbb{B}(x; \eta)$.
\end{definition}

\begin{definition}[Proximal normal cone]
\label{def:proximal_normal_cone}
Let $S \subset H$ be a closed set and $x \in S$. The proximal normal cone to $S$ at $x$ is
$$N^P(S; x) := \partial_P I_S(x)$$
where $I_S$ is the indicator function of $S$.
\end{definition}

\begin{lemma}[Proximal normal cone characterization]
\label{lem:proximal_normal_characterization}
\lean{MoreauSweeping.proximal_normal_characterization}
For any set $S \subset H$ and $x \in S$,
$$d_S(x) = 0.$$
\end{lemma}

\section{Approximate Projections}

\begin{definition}[Approximate projection]
\label{def:approximate_projection}
Let $S \subset H$ be a closed set, $\varepsilon > 0$ and $x \in H$. We define the set of $\varepsilon$-approximate projections as
$$\text{proj}_S^\varepsilon(x) := \left\{ z \in S : \|x - z\|^2 < d_S^2(x) + \varepsilon \right\}$$
\end{definition}

\begin{lemma}
\label{lem:approximate_projection_formula}
\lean{MoreauSweeping.approximate_projection_formula}
Let $S \subset H$, $x \in S$, and $\varepsilon > 0$. Then
$$x \in \text{proj}_S^\varepsilon(x).$$
\end{lemma}
