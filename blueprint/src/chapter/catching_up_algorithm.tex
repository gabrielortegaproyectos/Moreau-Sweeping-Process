\chapter{Catching-Up Algorithm with Errors}

This chapter presents the enhanced catching-up algorithm using approximate projections and establishes its main properties.

\section{The Sweeping Process}

\begin{definition}[Moreau's sweeping process]
\label{def:sweeping_process}
\uses{def:clarke_normal_cone}
Given a Hilbert space $H$ and a family of closed moving sets $(C(t))_{t\in[0,T]}$, Moreau's sweeping process is the differential inclusion
\begin{equation}\label{eq:sweeping_process}
\begin{cases}
\dot{x}(t) \in -N(C(t); x(t)) & \text{a.e. } t \in [0, T], \\
x(0) = x_0 \in C(0).
\end{cases}
\end{equation}
\end{definition}

\section{Classical Catching-Up Algorithm}

The classical catching-up algorithm, developed by J.J. Moreau for convex moving sets, consists of taking a time discretization $\{t_k^n\}_{k=0}^n$ of the interval $[0, T]$ and defining a piecewise linear continuous function $x^n : [0, T] \to H$ with nodes
$$x_{k+1}^n := \text{proj}_{C(t_{k+1}^n)}(x_k^n) \quad \text{for all } k \in \{0, \dots, n-1\}$$

Under suitable assumptions, the sequence $(x^n)$ converges to the unique solution of \eqref{eq:sweeping_process}.

\section{Enhanced Algorithm with Approximate Projections}

\begin{definition}[Catching-up algorithm with approximate projections]
\label{def:catching_up_algorithm}
\uses{eq:sweeping_process,def:approximate_projection,lem:approx_projection_nonempty}
Given a time discretization $\{t_k^n\}_{k=0}^n$ of $[0,T]$ and a sequence $(\varepsilon_k^n)$ of positive numbers, the catching-up algorithm with errors defines a piecewise linear function $x^n: [0,T] \to H$ with nodes
$$x_{k+1}^n \in \text{proj}_{C(t_{k+1}^n)}^{\varepsilon_k^n}(x_k^n)$$
for all $k \in \{0, \ldots, n-1\}$, with $x_0^n = x_0$.
\end{definition}

\begin{definition}[Admissible error schedule]
\label{def:admissible_error}
\lean{MoreauSweeping.AdmissibleError}
A sequence $(\varepsilon_k)$ is admissible if $\varepsilon_k > 0$ for every $k$.
\end{definition}

\begin{lemma}[Self-membership under admissible errors]
\label{lem:self_mem_approx_proj}
\lean{MoreauSweeping.self_mem_approxProj_of_admissibleError}
\uses{def:admissible_error, lem:approximate_projection_formula}
If $x \in S$ and $(\varepsilon_k)$ is admissible, then for every $k$:
$$x \in \text{proj}_{S}^{\varepsilon_k}(x).$$
\end{lemma}

\section{Main Properties}

\begin{theorem}[Properties of the algorithm]
\label{thm:algorithm_properties}
\uses{def:catching_up_algorithm,eq:sweeping_process,def:hausdorff,lem:approx_projection_mono}
\lean{MoreauSweeping.algorithm_properties}
\uses{def:catching_up_algorithm, def:approximate_projection, def:admissible_error, lem:self_mem_approx_proj}
Under suitable assumptions on the moving set $C$ and the error sequence $(\varepsilon_k^n)$, the sequence $(x^n)$ of functions generated by the catching-up algorithm satisfies certain boundedness and convergence properties.
\end{theorem}
